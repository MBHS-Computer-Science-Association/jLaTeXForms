\documentclass[letterpaper, 10pt]{article}
\usepackage[margin=1in]{geometry}
\begin{document}
    \setcounter{secnumdepth}{-1}

    Thai Nguyen

    Mr. Scott Martin

    AP World History GT - 1

    21 May 2016


    \part{1450 C.E. -- 1750 C.E.}
        \section{European monarchs versus Asian monarchs}
            \subsection{Similarities}
                \begin{itemize}
                    \item European monarchs and Asian monarchs were similar in ultimate role in government because they exerted power. European monarchs had feudal authority, and similarly Asian monarchs had authority over the bureaucracy. This key similarity between the two can be wholeheartedly explained because Europe was developing the feudal system and Confucianism influenced China.
                    \item European monarchs and Asian monarchs were similar in eventual decline because they lost power over time. European monarchs had lost authority to the churches, and similarly Asian monarchs had lost power to next monarch. This key similarity between the two can be wholeheartedly explained because because power was never in steady supply.
                    \item European monarchs and Asian monarchs were similar in effects because they encouraged artistic movements. European monarchs had Italian doges as patrons, and similarly Asian monarchs had been patrons of art in Japan. This key similarity between the two can be wholeheartedly explained because fancy people liked fancy things.
                    \item European monarchs and Asian monarchs were similar in manipulation because they used religion to help them. European monarchs had the supporting church, and similarly Asian monarchs had the mandate of heaven. This key similarity between the two can be wholeheartedly explained because religion was something in which many people believed.
                \end{itemize}
            \subsection{Differences}
                \begin{itemize}
                    \item European monarchs and Asian monarchs were different in emphasis because they exerted power differently. European monarchs had used Catholic Church, and instead Asian monarchs had used military shoguns. This striking juxtaposition can be explained by the awesome fact because Christianity was powerful since time of Rome.
                    \item European monarchs and Asian monarchs were different in degree of power because of limited governments. European monarchs had lessened power by the people, and instead Asian monarchs had only lost power to others. This striking juxtaposition can be explained by the awesome fact because the enlightenment was hitting European monarchs.
                \end{itemize}
        \section{African empires versus Asian and European empires}
            \subsection{Similarities}
                \begin{itemize}
                    \item African empires and Asian and European empires were similar in connection in trade. African empires had the trans-saharan trade route, and similarly Asian and European empires had the Indian Ocean. This key similarity between the two can be wholeheartedly explained because most empires formed from power derived from trade.
                    \item African empires and Asian and European empires were similar in government and rulers. African empires had rulers to govern people, and similarly Asian and European empires had monarchs to keep order. This key similarity between the two can be wholeheartedly explained because stateless societies were not as effective at maintaining whole empires.
                    \item African empires and Asian and European empires were similar in economic integration. African empires had their economy integrated through trade of gold and salt, and similarly Asian and European empires had their economy producing goods and items for trade. This key similarity between the two can be wholeheartedly explained because trade brought taxes for the empire.
                \end{itemize}
            \subsection{Differences}
                \begin{itemize}
                    \item African empires and Asian and European empires were different in style of ruling. African empires had a city-state style in West Africa, and instead Asian and European empires had a shogunate in Japan. This striking juxtaposition can be explained by the awesome fact because Africa had not experienced a unifying movement because of the different identities of tribes.
                    \item African empires and Asian and European empires were different in timing of demise. African empires had fallen to Islamic empires with the Umayyad Dynasty, and instead Asian and European empires had fallen to the Mongols. This striking juxtaposition can be explained by the awesome fact because no empire encompassed all of Afro-Eurasia at one time.
                    \item African empires and Asian and European empires were different in foreign policy. African empires had Mansa Musa who made hajj and became well-known and was connected with the world, and instead Asian and European empires had Chinese taking a policy of less outward connection. This striking juxtaposition can be explained by the awesome fact because different cultural influences internally.
                \end{itemize}
        \section{Slavery versus Serfdom}
            \subsection{Similarities}
                \begin{itemize}
                    \item Slavery and Serfdom were similar in labor in which they were involved because of the need for human labor. Slavery had to work in agriculture, and similarly Serfdom had to work on manors. This key similarity between the two can be wholeheartedly explained because everywhere rich people needed human labor because of lack of machinery.
                    \item Slavery and Serfdom were similar in owners because of differing social class. Slavery had been owned by middle class people, and similarly Serfdom had was for people in manors and large property. This key similarity between the two can be wholeheartedly explained because land was a leveraging factor in owning serfs because of debts.
                    \item Slavery and Serfdom were similar in opportunity. Slavery had been able to work off servitude, and similarly Serfdom had been able to work off servitude. This key similarity between the two can be wholeheartedly explained because it was customary to do so in some places.
                    \item Slavery and Serfdom were similar in geographical extent because both were extensive everywhere. Slavery had been practiced in Americas, and similarly Serfdom had been practiced in Europe and Russia. This key similarity between the two can be wholeheartedly explained because people need people to do stuff for them.
                \end{itemize}
            \subsection{Differences}
                \begin{itemize}
                    \item Slavery and Serfdom were different in subjects of servitude. Slavery had American slavery consisting of Africans and natives, and instead Serfdom had Russian serfdom consisting of local populations. This striking juxtaposition can be explained by the awesome fact because Americas needed people imported to replace dead natives.
                    \item Slavery and Serfdom were different in locations. Slavery had been in Americas mainly, and instead Serfdom had been in Russia and Europe mainly. This striking juxtaposition can be explained by the awesome fact because Americas needed slave imports to replace dead native populations.
                \end{itemize}
        \section{Russian-Western interaction versus Ottoman-Western interaction}
            \subsection{Similarities}
                \begin{itemize}
                    \item Russian-Western interaction and Ottoman-Western interaction were similar in cultural influence. Russian-Western interaction had Peter the Great build St. Petersburg with a window on the West, and similarly Ottoman-Western interaction had to deal with the nationalistic Western states. This key similarity between the two can be wholeheartedly explained because the West was having its moment in history now.
                    \item Russian-Western interaction and Ottoman-Western interaction were similar in threatened by growing strength of west. Russian-Western interaction had became aware of the importance of industrialism, and similarly Ottoman-Western interaction had become known as the sick man of Europe. This key similarity between the two can be wholeheartedly explained because they were not experiencing industrialization.
                \end{itemize}
            \subsection{Differences}
                \begin{itemize}
                    \item Russian-Western interaction and Ottoman-Western interaction were different in openness to culture. Russian-Western interaction had been very open to the West, and instead Ottoman-Western interaction had been not very open to modernization. This striking juxtaposition can be explained by the awesome fact because Islam sometimes found Westernization against their ideals.
                    \item Russian-Western interaction and Ottoman-Western interaction were different in end result. Russian-Western interaction had ended with Russia become an enemy of the west, and instead Ottoman-Western interaction had ended with Ottomans being crushed. This striking juxtaposition can be explained by the awesome fact because Russia was able to adopt their communistic ways.
                \end{itemize}
        \section{Transatlantic trade versus Indian Ocean trade}
            \subsection{Similarities}
                \begin{itemize}
                    \item Transatlantic trade and Indian Ocean trade were similar in had transported goods in bulk. Transatlantic trade had transported food and slaves, and similarly Indian Ocean trade had transported spices and silver. This key similarity between the two can be wholeheartedly explained because ocean-going ships could carry a large amount of items.
                    \item Transatlantic trade and Indian Ocean trade were similar in wealth to be sought. Transatlantic trade had a wealthy slave trade, and similarly Indian Ocean trade had lucrative silver trade with China. This key similarity between the two can be wholeheartedly explained because trade with other regions was beneficial to everyone else.
                    \item Transatlantic trade and Indian Ocean trade were similar in transport of culture as well. Transatlantic trade had mixed African and Christian religious traditions, and similarly Indian Ocean trade had mixed Muslim and Buddhist traditions in southeast Asia. This key similarity between the two can be wholeheartedly explained because culture was always in motion.
                    \item Transatlantic trade and Indian Ocean trade were similar in integrated world. Transatlantic trade had added America to the global trade, and similarly Indian Ocean trade had connected Asia and Europe to the global trade. This key similarity between the two can be wholeheartedly explained because they connected far away lands by ocean.
                \end{itemize}
            \subsection{Differences}
                \begin{itemize}
                    \item Transatlantic trade and Indian Ocean trade were different in control of trade. Transatlantic trade had been controlled by Spain and Portugal, and instead Indian Ocean trade had been controlled by Muslims. This striking juxtaposition can be explained by the awesome fact because Spain and Portugal were connected directly to the Atlantic.
                    \item Transatlantic trade and Indian Ocean trade were different in geography. Transatlantic trade had no monsoons, and instead Indian Ocean trade had monsoons which had to be learned based on the seasons. This striking juxtaposition can be explained by the awesome fact because Indian Ocean was mainly in the tropics.
                \end{itemize}
    \part{1750 C.E. -- 1914 C.E.}
        \section{Industrial Revolution in Europe, Russia, and Japan}
            \subsection{Similarities}
                \begin{itemize}
                    \item Industrial Revolution in Europe and Industrial Revolution in Japan were similar in effects on economy. Industrial Revolution in Europe had grown the economy and power, and similarly Industrial Revolution in Japan had risen Japan to higher status of empire. This key similarity between the two can be wholeheartedly explained because tools of warfare opened doors.
                    \item Industrial Revolution in Japan and Industrial Revolution in Russia were similar in militarization. Industrial Revolution in Japan had allowed Japan to invade China, and similarly Industrial Revolution in Russia had allowed Russia to gain other states. This key similarity between the two can be wholeheartedly explained because weapons are made industrially.
                \end{itemize}
            \subsection{Differences}
                \begin{itemize}
                    \item Industrial Revolution in Japan and Industrial Revolution in Europe were different in timing. Industrial Revolution in Japan had started as a reaction to Western industrialization, and instead Industrial Revolution in Europe had started independently in Britain. This striking juxtaposition can be explained by the awesome fact because Britain had the resources.
                    \item Industrial Revolution in Japan and Industrial Revolution in Russia were different in expansion. Industrial Revolution in Japan had been able to invade southeast asia, and instead Industrial Revolution in Russia had added parts of Asia to its empire. This striking juxtaposition can be explained by the awesome fact because Japan had an outlet with the ocean.
                \end{itemize}
        \section{Italian nationalism versus German nationalism}
            \subsection{Similarities}
                \begin{itemize}
                    \item Italian nationalism and German nationalism were similar in belief in racial superiority. Italian nationalism had wanted to promote the Italian nation, and similarly German nationalism had wanted to promote German nation. This key similarity between the two can be wholeheartedly explained because similar languages and cultures among the region.
                    \item Italian nationalism and German nationalism were similar in their origins because they were results of a geographical unification. Italian nationalism had several city states unified, and similarly German nationalism had different kingdoms united. This key similarity between the two can be wholeheartedly explained because sought to fight against other nations.
                    \item Italian nationalism and German nationalism were similar in led to participation in the world wars. Italian nationalism had fought in both wars, and similarly German nationalism had fought in both wars. This key similarity between the two can be wholeheartedly explained because they had an element of militarism added.
                    \item Italian nationalism and German nationalism were similar in causes because they believed in their superior culture. Italian nationalism had elite traditional Italian culture, and similarly German nationalism had "Aryan" German culture. This key similarity between the two can be wholeheartedly explained because wanted to bond their society together more effectively.
                \end{itemize}
            \subsection{Differences}
                \begin{itemize}
                    \item Italian nationalism and German nationalism were different in effects because German nationalism was stronger. Italian nationalism had the persecution of non-Italians, and instead German nationalism had resulted in the holocaust. This striking juxtaposition can be explained by the awesome fact because Germans were under the instruction of Adolf Hitler.
                    \item Italian nationalism and German nationalism were different in resulting political groups because of the different forms of fascism. Italian nationalism had the Black Hand under Mussolini, and instead German nationalism had the Nazi party under Hitler. This striking juxtaposition can be explained by the awesome fact because Hitler was much more forceful.
                    \item Italian nationalism and German nationalism were different in effects because Germany became more industrialized. Italian nationalism had not become as powerful, and instead German nationalism had became very powerful. This striking juxtaposition can be explained by the awesome fact because Germany was a result of many kingdoms combining.
                \end{itemize}
        \section{African imperialism versus Indian imperialism}
            \subsection{Similarities}
                \begin{itemize}
                    \item African imperialism and Indian imperialism were similar in mother country. African imperialism had Britain was the mother country, and similarly Indian imperialism had Britain was the mother country. This key similarity between the two can be wholeheartedly explained because Britain owned everything.
                    \item African imperialism and Indian imperialism were similar in treatment of subjects because both suffered. African imperialism had made natives work for free, and similarly Indian imperialism had made Indians not have freedoms. This key similarity between the two can be wholeheartedly explained because Europeans wanted to extract wealth from them.
                \end{itemize}
            \subsection{Differences}
                \begin{itemize}
                    \item African imperialism and Indian imperialism were different in time of release. African imperialism had ended early by the British ceding control to Afrikaaners, and instead Indian imperialism had ended later. This striking juxtaposition can be explained by the awesome fact because Indians had to have their Indian National Congress to petition first.
                    \item African imperialism and Indian imperialism were different in relative violence. African imperialism had lots of violence during apartheid, and instead Indian imperialism had lots of violence with partition. This striking juxtaposition can be explained by the awesome fact because people don't like people who are different.
                    \item African imperialism and Indian imperialism were different in source of conflict. African imperialism had internal conflict with white afrikaaners, and instead Indian imperialism had external conflict with Britain. This striking juxtaposition can be explained by the awesome fact because the dutch were in South Africa.
                \end{itemize}
        \section{European women in higher classes versus European women in lower classes}
            \subsection{Similarities}
                \begin{itemize}
                    \item European women in higher classes and European women in lower classes were similar in same sexism. European women in higher classes had not been able to have freedom outside of the sphere, and similarly European women in lower classes had not been able to go make money in prestigious jobs. This key similarity between the two can be wholeheartedly explained because the roles of women were set.
                    \item European women in higher classes and European women in lower classes were similar in treatment because they had the same perspective. European women in higher classes had been treated as pretty objects, and similarly European women in lower classes had been treated as property. This key similarity between the two can be wholeheartedly explained because the strong patriarchy.
                \end{itemize}
            \subsection{Differences}
                \begin{itemize}
                    \item European women in higher classes and European women in lower classes were different in wealth. European women in higher classes had more money and more jewels, and instead European women in lower classes had less money and less jewels. This striking juxtaposition can be explained by the awesome fact because higher class means more money.
                    \item European women in higher classes and European women in lower classes were different in abilities. European women in higher classes had the option to become advisors to kings, and instead European women in lower classes had pretty much nothing. This striking juxtaposition can be explained by the awesome fact because royalty was a large door-opener.
                \end{itemize}
\end{document}
